% !TEX program = xelatex
\documentclass[compress,aspectratio=169]{beamer}

\usepackage[english]{babel}
\usepackage{metalogo}
\usepackage{listings}
\usepackage{fontspec}
\usepackage{unicode-math}
\usepackage[theme=nord,charsperline=60,linenumbers]{clisting}

\usetheme{Nord}

\setmainfont{Roboto}
\setmonofont{JuliaMono}

\title{Customized Nord Beamer Theme}
\subtitle{A simple beamer theme that uses "Nord" color scheme}
\author{Original by Junwei Wang \texorpdfstring{\\ Customized by Simon A. Bjørn}{}}
\institute{University of Oslo}
\date{11.01.2024}

\begin{document}

\begin{frame}[plain,noframenumbering]
    \maketitle
\end{frame}

\begin{frame}
    \frametitle{Usage}
    This is an edited version of the original theme by Junwei Wang. The original theme can be found at \url{https://github.com/junwei-wang/beamerthemeNord}.

    \vspace{1em}

    The changes done to the theme are settings I wanted to have for my own presentations. These changes are:
    \begin{itemize}
        \item Added a '{\textbackslash}pin'-command to be able to create bullets (\pin) outside of itemize environments
        \item Added Nord-styled listings for any language defined in the listings package and Julia. \\
        Inline listings with the \texttt{{\textbackslash}cinl\{lang\}\{...code...\}}. For Julia, use \texttt{{\textbackslash}jlinl\{...code...\}} \\
        Block listings with the \texttt{clisting} environment. For Julia, use \texttt{jllisting}.
        \item Changed text color in blocks to Nord's white color for better contrast. See slide 3.
            
    \end{itemize}
\end{frame}

\begin{frame}[fragile]
    \frametitle{Code listings}
    \begin{columns}[t]
        \begin{column}{0.5\textwidth}
            Julia specific: \\
            Inline: \jlinl{println("Hello World!")} \\
            Block:  
            \begin{jllisting}[gobble=16]
                using LinearAlgebra

                A = rand(3, 3)
                println(eigvals(A))
            \end{jllisting}
        \end{column}
        \begin{column}{0.5\textwidth}
            Any other language from listings \\
            Inline: \cinl{c++}{std::cout << "Hello World!" << std::endl;} \\
            Block:
            \begin{clisting}[language=c++, gobble=16]
                #include <iostream>

                int main() {
                    std::cout << "Hello World!" << std::endl;
                    return 0;
                }
            \end{clisting}
            
        \end{column}
    \end{columns}
\end{frame}

\begin{frame}
  \frametitle{Blocks}
  \begin{block}{This is a Block}
    \[
      a^2 + b^2 = c^2
    \]
  \end{block}
  \begin{exampleblock}{This is an Example Block}
    \[
      E = m \cdot c^{2}
    \]
  \end{exampleblock}
  \begin{alertblock}{This is an Alert Block}
    \[
      e^{i\pi} + 1 = 0
    \]
  \end{alertblock}

  \centering
  \begin{minipage}{1.0\linewidth}
    \begin{block}{Horizontally-Aligned Block}
      \[
        \log xy = \log x + \log y
      \]
    \end{block}
  \end{minipage}
\end{frame}

\end{document}
